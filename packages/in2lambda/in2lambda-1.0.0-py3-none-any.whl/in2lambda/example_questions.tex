\documentclass[12pt]{article}

% Mechanics course definitions and macros.
%\usepackage{mechanics}
\usepackage{graphicx}

% Add space between paragraphs instead of indenting.
\usepackage{parskip}

\addtolength{\topmargin}{-4em}
\addtolength{\evensidemargin}{-3em}
\addtolength{\oddsidemargin}{-3em}
\addtolength{\textheight}{6em}
\addtolength{\textwidth}{4em}

%\usepackage{bm}
\usepackage{graphics}
\usepackage{color}
\usepackage{amsmath}

\begin{document}
\section*{\textrm{\textbf{Mechanics Problem Sheet 4 --- Collisions and CoM frame}}} 

\begin{enumerate}
%\begin{enumerate}
\item The ballistic pendulum is a device for determining the speed of a
  projectile. A large stationary mass $M$ is suspended by a wire from
  the ceiling and a projectile of mass $m$ is fired into it at an
  unknown speed $u$. The projectile embeds itself in the hanging mass.
  Assume that $g = 9.8\,\text{m}\cdot\text{s}^{-2}$.
  \begin{enumerate}
  \item Is momentum conserved in the collision? Is kinetic energy
    conserved?
  \item Derive an expression for the height $h$ to which the masses rise
    after the collision.
  \item A bullet of mass $m = 12\,$g is fired into a block of mass
    $M = 10\,$kg and the height rise $h$ is measured to be $10\,$cm.
    What was the speed of the bullet?
  \end{enumerate}

\item A ping-pong (table tennis) ball makes a head-on elastic collision
  with a much heavier basketball, which is initially stationary. Which 
  of the following statements are true after the collision?
  \begin{enumerate}
  \item The absolute value of the momentum of the ping-pong ball is
    smaller than that of the basketball.
  \item The absolute value of the momentum of the ping-pong ball is the
    same as that of the basketball.
  \item The absolute value of the momentum of the ping-pong ball is
    greater than that of the basketball.
  \item Not enough information is given to decide between (a), (b) and
    (c).
  \item The ping-pong ball has less kinetic energy than the basketball.
  \item The ping-pong ball has the same kinetic energy as the
    basketball.
  \item The ping-pong ball has more kinetic energy than the basketball.
  \item Not enough information is given to decide between (e), (f) and
    (g).
  \end{enumerate}

\item Question 3 on Problem Sheet 3 (repeated below) was about a woman
  walking along a plank on a frictionless frozen lake. Solve this
  problem again, this time using the concept of the centre of mass.

%  \begin{tcolbox}{0.96\textwidth}
%  {Woman walking along a plank on a frozen lake} 
    A woman of mass 60\,kg stands at one end of a 10\,m plank of
    mass 20\,kg, which itself lies on a (frictionless) frozen lake. She
    walks to the other end of the plank. Work out how far she has
    travelled relative to the lake.
 % \end{tcolbox}

\item Two masses, $m_1 = 4\,$kg and $m_2 = 6\,$kg, are moving in the
  $x$-$y$ plane. The position of $m_1$ is given by
  $\vec{\boldsymbol{r}}_1(t) = (1 + 3t)\boldsymbol{\hat{i}} + t\boldsymbol{\hat{j}}$ and the position of $m_2$ by
  $\vec{\boldsymbol{r}}_2(t) = -2t\boldsymbol{\hat{i}} + 5\boldsymbol{\hat{j}}$. Positions are measured in metres and
  the time $t$ is measured in seconds.
  \begin{enumerate}
  \item What is the reduced mass of the system?
  \item What is the velocity of the centre of mass?
  \item What is the velocity of $m_1$ relative to $m_2$?
  \item What is the momentum of $m_1$ in the centre-of-mass frame?
  \item What is the total kinetic energy of the system in the
    centre-of-mass frame?
  \end{enumerate}

\item The position $\vec{\boldsymbol{R}}(t)$ of the centre of mass of $N$ objects with
  masses $m_i$ at positions $\vec{\boldsymbol{r}}_i(t)$ is given by:
  %
  \begin{displaymath}
    \vec{\boldsymbol{R}}(t) = \frac{\sum_{i=1}^{N} m_i \vec{\boldsymbol{r}}_i(t)}{\sum_{i=1}^{N} m_i} .
  \end{displaymath}
  %
  Show that the vector sum of the momenta of all of the masses is zero
  in the centre-of-mass frame. In other words, show that the
  centre-of-mass frame is also the zero-momentum frame.
  
\item In lectures, you saw that the total laboratory-frame kinetic
  energy of two particles is the sum of their kinetic energy in the
  centre-of-mass frame and the kinetic energy of the centre of mass in
  the laboratory frame. Show that the same result holds for a system of
  $N$ particles. You may assume (because you proved it in the previous question) that the
  total momentum of the $N$ particles is zero in the centre-of-mass
  frame.

\item In one-dimensional inelastic collisions, the coefficient of
  restitution $e$ is defined by the equation
  % 
  \begin{displaymath}
    v_2 - v_1 = -e (u_2 - u_1),
  \end{displaymath}
  % 
  where $v_1$ and $v_2$ are the final velocities of the two bodies and
  $u_1$ and $u_2$ are their initial velocities. Use this equation, plus
  momentum conservation, to derive the following expressions for the
  final velocities of the two bodies (of masses $m_1$ and $m_2$) in the
  case when the second body is initially at rest ($u_2 = 0$):
  % 
  \begin{align*}
    v_1 &= \frac{(m_1 - e m_2) u_1}{m_1 + m_2},
    &
      v_2 &= \frac{(1 + e)m_1 u_1}{m_1 + m_2}.
  \end{align*}



%\end{enumerate}


%\subsection*{Problems for Reflection and Discussion}


%\begin{enumerate}
\item 
  A ball is dropped onto a table from an initial height $h_0$. If it
  lands with speed $v$, it bounces up with speed $ev$, where the
  coefficient of restitution $e$ is a positive constant less than 1.
  %
  \begin{enumerate}
  \item Write down expressions for the speed $v_0$ of the ball when it
    first hits the table and the time $t_0$ it takes to drop.
    
  \item Find the height $h_1$ reached by the ball after its first bounce
    and the corresponding drop time $t_1$. Write down expressions for
    the height $h_n$ and drop time $t_n$ of the $n^{\text{th}}$ bounce.
  \item How long does the ball take to stop completely? Evaluate this
    time if the initial height $h_0 = 5\,$m and the coefficient of
    restitution $e = 0.5$. Take $g = 10\,\text{m}\cdot\text{s}^{-2}$.

    [Hint: you will need the formula for the sum of a geometric series:

    \centerline{$\displaystyle a + ar + ar^2 + ar^3 + \ldots =
      \frac{a}{1-r}, \qquad |r|<1.]$}
  \end{enumerate}
  
\item In the chemical reaction $\text{NO} + \text{O}_3 \rightarrow
  \text{NO}_2 + \text{O}_2$, there is an intermediate step where the
  $\text{NO}$ and $\text{O}_3$ molecules combine to form an activated
  complex $[\text{NO--O}_3]^{\ast}$. The diagram below shows an
  empirical potential energy curve representing the three different
  stages of the reaction.
  %
  \begin{center}
    \includegraphics[width=0.75\textwidth]{figures/activated_complex.png}
  \end{center}
  %
  Although the $x$ axis corresponds to an unspecified reaction
  coordinate, the diagram can be interpreted as any other potential
  function. The peak of the potential curve is 9.6\,kJ$\cdot$ mol$^{-1}$
  above the combined potential energy of the $\text{NO} + \text{O}_3$
  reactants; this is the ``activation energy'' required for the reaction
  to take place. One mole consists of Avogadro's number, $6.022 \times
  10^{23}$, of $[\text{NO--O}_3]^{\ast}$ complexes. The mass of a
  nitrogen atom is 14 atomic mass units and the mass of an oxygen
  atom is 16 atomic mass units.

  \smallskip
  
  An $\text{NO}$ molecule travelling in the $x$ direction at speed
  $u_{\text{NO}} = -550\,\text{m}\cdot\text{s}^{-1}$ and an $\text{O}_3$
  molecule travelling at
  $u_{\text{O}_3} = +550\,\text{m}\cdot\text{s}^{-1}$ collide. After the
  collision, the molecules temporarily coalesce to form an
  $[\text{NO--O}_3]^{\ast}$ complex.
  \begin{enumerate}
  \item Find the velocity of the $[\text{NO--O}_3]^{\ast}$ complex
    after the collision.
  \item Show that the collision is inelastic. What has happened to the
    kinetic energy lost in the collision?
  \item Will this collision lead to the formation of $\text{NO}_2$?
    Explain your answer.
  \item Two oxygen atoms collide in isolation. Could the collision
    result in the formation of an $\text{O}_2$ molecule in the right
    circumstances?
  \end{enumerate}
  

\item Show that the velocity $v(t)$ of a rocket starting from rest and
  accelerating against the Earth's gravity is given by
  %
  \begin{displaymath}
    v(t) = u\ln \left ( \frac{M_0}{M(t)} \right ) - gt.
  \end{displaymath}
  The exhaust speed is $u$, the initial mass is $M_0$, and the mass
  after time $t$ is $M(t)$. Neglect the dependence of $g$ on height.

\item A two-stage rocket accelerates in free space by ejecting fuel at a
  constant relative speed $u$. Three-quarters of its initial mass is
  fuel. It accelerates from rest until $2/3$ of the fuel is burnt. The
  first stage is then detached. Fuel accounts for $4/5$ of the mass of
  the second stage. Show that the final speed of the second stage is
  equal to $u\ln(10)$.

  Show that a single-stage rocket of the same initial mass, burning 
  the same total mass of fuel, would reach a final speed of 
  approximately $0.6u\ln(10)$.

\end{enumerate}


  
%\subsubsection*{Numerical Answers}
%
%\begin{list}{}{\setlength{\itemsep}{0em}\setlength{\labelwidth}{6em}}
%\item[1.] 2.5\,m.
%\item[2.] (a) 2.4\,kg; (b) $0.4\boldsymbol{\hat{j}}\,$m\udot s$^{-1}$; (c) $(5\boldsymbol{\hat{i}} +
%  \boldsymbol{\hat{j}})\,$m\udot s$^{-1}$; (d) $(12\boldsymbol{\hat{i}} + 2.4\boldsymbol{\hat{j}})\,$kg\udot m\udot
%  s$^{-1}$; (e) 31.2\,J.
%\end{list}

\end{document}