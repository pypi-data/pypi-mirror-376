% arguelles v2.4.2
% author: Michele Piazzai
% https://github.com/piazzai
% license: MIT

\documentclass[lualatex,compress,12pt]{beamer}
\usepackage{fontspec}
\usepackage{braket}
\usepackage[backend=biber,style=authoryear]{biblatex}
\addbibresource{qubit.bib}
\usepackage{svg}


\setmainfont{EB Garamond}
\setbeamertemplate{caption}{\raggedright\insertcaption\par}


\usetheme{Antibes}
\usefonttheme{serif}
\usecolortheme{spruce}


\title{On Qubit Routing}
\subtitle{in Quantum Computer Architectures}

\date{\today}
\author{ M. Sobolev
	\\ {\small Supervised by: L. Laneve \and W. Schober  \and Prof. S. Wolf}}
\institute{USI}


\begin{document}
	
	\frame[plain]{\titlepage}

	


	

\section{Problem statement}
\begin{frame}
	\frametitle{Universal Quantum Computer}
	
	\begin{figure}
		\includegraphics[height=3cm]{pic/DavidDeutsch_2009G-stageshot.jpg}
		\caption{David Deutsch}
	\end{figure}
	
	\begin{definition}
		 The tape (\(t\)) of Deutsch’s Turing machine \(\mathfrak{U}\) consists of an infinite sequence of qubits and its finite control consists of a finite sequence of qubits (\(m\)). <...> \parencite{gruskaQuantumComputing2000} 
	\end{definition}
	


\end{frame}


\begin{frame}
	\frametitle{Circuit Model}
	
		\begin{figure}
		\includegraphics[height=3cm]{pic/deutsch algorithm.png}
		\caption{Circuit for Deutsch-Josza algorithm \parencite{nielsenQuantumComputationQuantum2010}}
	\end{figure}
\end{frame}


\begin{frame}
	\frametitle{Two-Qubit Gates}
	
		\begin{figure}
		\includegraphics[height=4cm]{pic/cnot.png}
		\caption{A CNOT gate \parencite{nielsenQuantumComputationQuantum2010}}
	\end{figure}
	
\end{frame}

\begin{frame}
	\frametitle{Quantum hardware connectivity}
	
	\begin{figure}
		\includesvg[scale=0.3,inkscapelatex=false,pretex=\tiny]{pic/topology}
		\caption{IBM  Brisbane connectivity \parencite{IBMQuantumHeavy}}
	\end{figure}
	
\end{frame}

\begin{frame}
	\frametitle{SWAP Gate}
	
	A swap gate SWAPs two qubit states, "moving" the state along the physical entities.
	\[
	\text{SWAP} (|\phi_{1}\rangle \otimes |\phi_{2}\rangle) = |\phi_{2}\rangle \otimes |\phi_{1}\rangle.
	\]
	\begin{figure}
		\includegraphics[height=4cm]{pic/swap-gate-symbol.png}
		\caption{A SWAP gate symbol \parencite{bergholmPennyLaneAutomaticDifferentiation2022}}
	\end{figure}
	
\end{frame}



\section{Solving considerations}

\begin{frame}
	\frametitle{Qubit routing}
	

	We'd like to execute the circuit on a given hardware topology
	\begin{columns}[b]
		% create the column with the first image, that occupies
		% half of the slide
		\begin{column}{.5\textwidth}
			\begin{figure}
				\centering
				\includegraphics[height=4cm]{pic/cowtan topology.png}
				\caption{Hardware topology}
			\end{figure}      
		\end{column}
		% create the column with the second image, that also
		% occupies half of the slide
		\begin{column}{.5\textwidth}
			\begin{figure}
				\centering
				\includegraphics[height=3cm]{pic/cowtan routing.png}
				\caption{Routed circuit \parencite{cowtanQubitRoutingProblem2019}}
			\end{figure}
		\end{column}
	\end{columns}
	


	
	
	
\end{frame}	


\begin{frame}
	\frametitle{NP hardness}
	
	
	Subgraph isomorphism (initial mapping) and token swapping (qubit routing)
	\begin{columns}[b]
		% create the column with the first image, that occupies
		% half of the slide
		\begin{column}{.5\textwidth}
			\begin{figure}
				\centering
				\includegraphics[height=3cm]{pic/F83A0F35-7911-4617-8D12-97A57410B5EB-low.png}
				\tiny
				\caption{ Graph isomorphism}
			\end{figure}      
		\end{column}
		% create the column with the second image, that also
		% occupies half of the slide
		\begin{column}{.5\textwidth}
			\begin{figure}
				\centering
				\includegraphics[height=3cm]{pic/Screenshot\_20250522\_081719.png}
				\tiny
				\caption{ Token swapping \parencite{yamanakaSwappingLabeledTokens2015}}
			\end{figure}
		\end{column}
	\end{columns}
	
\end{frame}	

\begin{frame}
	\frametitle{Example solution \parencite{zulehnerEfficientMethodologyMapping2018}}
	
	
	\begin{enumerate}
	
	\item Slicing the circuit into \textbf{incompatible} timesteps
	
	\item For every timestep:
\begin{enumerate}
		\item Getting all the hardware compliant permutations
		\item Use \(A^*\) algorithm for finding the way to the compliant permutation from the starting point
	\end{enumerate}

	
	\end{enumerate}
	
	\begin{figure}
		\centering
		\includegraphics[height=3cm]{pic/a star.png}
		\tiny
		\caption{ \(A*\) search algorithm }
	\end{figure}
	
\end{frame}	

\section{Conclusion}


\begin{frame}
	\frametitle{Possible outcomes}
	
	
	\begin{itemize}
		
		\item Reduce the qubit routing problem to a known NP hard problem 
		\item Derive an approximation algorithm with a known ratio
		\item Analyse the influence of the hardware graph characteristics to the hardness of the solution
		\item And many more 
	\end{itemize}
	
\end{frame}	

\begin{frame}
	\frametitle{Wrap up}
	
	
	\begin{itemize}
		
		\item Qubit routing is a hardware problem, haunting the execution of the quantum circuits on the NISQ devices
		\item The problem is believed to be NP hard
		\item Biggest trend is to solve it heuristically, however the approximation algorithms are also possible
	\end{itemize}
	
\end{frame}	

\begin{frame}
  \centering \Large
	\emph{Thank you}

\end{frame}	

\begin{frame}[allowframebreaks]
	\frametitle{Bibliography}
	
	\printbibliography

\end{frame}


\end{document}