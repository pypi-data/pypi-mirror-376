% Some commands used in this file
\newcommand{\package}{\emph}

\chapter{Introduction}

Qubit routing is a problem which emerged with the rise of NISQ devices and is a result of quantum hardware restrictions. The go-to model for planning and development of quantum algorithms was created by Deutsch \cite{deutschQuantumComputationalNetworks1989}. Deutsch designed, having a quantum Turing machine in mind, a quantum circuit model, which is represented by a set of wires (each of them represents a quantum bit ), and a set of quantum gates, which are ordered execution-wise. Every quantum gate can employ one or more wires (without any location requirements — every combination of wires can participate in any gate). The only model restriction is that — qubit can’t be engaged in multiple operations, represented by quantum gates, at the same time. And from this unbounded representation comes the qubit routing problem — quantum hardware is not equivalent (and probably can’t) to the circuit model. In modern quantum hardware, qubits can interact in multi-qubit gates only with a fixed set of physical qubits. However unwanted, this problem could be dealt with by the SWAP gate insertion, which effectively swaps two qubits; and if we assume all-to-all qubit connectivity (more formal definition would be — if the hardware topology graph is represented by one completely connected component), then this whole problem is boiled down to the efficient swap gates insertion, that’s why sometimes “qubit routing” problem is called “swap insertion”. Our work aims to overview the field of qubit routing, to present a study of a problem, and to formulate a solution to it.