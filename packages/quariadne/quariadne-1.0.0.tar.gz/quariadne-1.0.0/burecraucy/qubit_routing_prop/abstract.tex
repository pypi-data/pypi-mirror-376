\begin{abstract}

The basis of all the modern quantum algorithms was outlined by Deutsch \cite{deutschQuantumComputationalNetworks1989}. It presented a modelling paradigm for all the quantum algorithms, later proven through the quantum Turing machine equivalence conjecture by Yao \cite{chi-chihyaoQuantumCircuitComplexity1993}. The creation of a quantum computer architecture nowadays is acting as a basic toolset for the projection and analysis of all the quantum computation algorithms. However, the realistic hardware architecture is highly different from the traditional computational model -- which is de facto assumed by Deutsch, i.e. von Neumann architecture \cite{vonneumannFirstDraftReport1993}. This departure in particular is happening because the qubits in the layout don't have the equivalent connectivity properties, i.e. not all the physical qubits in the circuit are made equal (see, for example, an informal discussion of IBM quantum architecture connectivity \cite{IBMQuantumHeavy}). Thus, an additional constraint satisfaction at the compilation step is required, termed "qubit routing". We will formalise the process of quantum circuit conversion to the qubit interaction graph, combining time- and spatial-related information. Then we aim to explore the possibilities of the optimal placement of CNOT gates (and potential insertion of SWAP gates) in the architecture-agnostic computation graph. After that, we will take the constraints of the different quantum computer architectures and adapt our routing/mapping algorithms to those, presenting an additional set of graphs. Finally, we will port the results of the routing to the main quantum circuit compilers, e.g. qiskit, and evaluate those on the main benchmarks for the selected quantum architectures.
\end{abstract}
