%% (Master) Thesis template
% Template version used: v1.4
%
% Largely adapted from Adrian Nievergelt's template for the ADPS
% (lecture notes) project.


%% We use the memoir class because it offers a many easy to use features.
\documentclass[11pt,a4paper,titlepage]{memoir}

%% Packages
%% ========

%% LaTeX Font encoding -- DO NOT CHANGE
\usepackage[OT1]{fontenc}

%% Babel provides support for languages.  'english' uses British
%% English hyphenation and text snippets like "Figure" and
%% "Theorem". Use the option 'ngerman' if your document is in German.
%% Use 'american' for American English.  Note that if you change this,
%% the next LaTeX run may show spurious errors.  Simply run it again.
%% If they persist, remove the .aux file and try again.
\usepackage[english]{babel}
\makeatletter\AtBeginDocument{\let\@elt\relax}\makeatother

%% Input encoding 'utf8'. In some cases you might need 'utf8x' for
%% extra symbols. Not all editors, especially on Windows, are UTF-8
%% capable, so you may want to use 'latin1' instead.
\usepackage[utf8]{inputenc}

%% This changes default fonts for both text and math mode to use Herman Zapfs
%% excellent Palatino font.  Do not change this.
\usepackage[sc]{mathpazo}

%% The AMS-LaTeX extensions for mathematical typesetting.  Do not
%% remove.
\usepackage{amsmath,amssymb,amsfonts,mathrsfs}

%% NTheorem is a reimplementation of the AMS Theorem package. This
%% will allow us to typeset theorems like examples, proofs and
%% similar.  Do not remove.
%% NOTE: Must be loaded AFTER amsmath, or the \qed placement will
%% break
\usepackage[amsmath,thmmarks]{ntheorem}

%% LaTeX' own graphics handling
\usepackage{graphicx}
\usepackage{rotating}


%% We unfortunately need this for the Rules chapter.  Remove it
%% afterwards; or at least NEVER use its underlining features.
\usepackage{soul}

%% This allows you to add .pdf files. It is used to add the
%% declaration of originality.
\usepackage{pdfpages}

%% Some more packages that you may want to use.  Have a look at the
%% file, and consult the package docs for each.
%% See the TeXed file for more explanations

%% [OPT] Multi-rowed cells in tabulars
%\usepackage{multirow}

%% [REC] Intelligent cross reference package. This allows for nice
%% combined references that include the reference and a hint to where
%% to look for it.
\usepackage{varioref}

%% [OPT] Easily changeable quotes with \enquote{Text}
%\usepackage[german=swiss]{csquotes}

%% [REC] Format dates and time depending on locale
\usepackage{datetime}

%% [OPT] Provides a \cancel{} command to stroke through mathematics.
%\usepackage{cancel}

%% [NEED] This allows for additional typesetting tools in mathmode.
%% See its excellent documentation.
\usepackage{mathtools}

%% [ADV] Conditional commands
%\usepackage{ifthen}

%% [OPT] Manual large braces or other delimiters.
%\usepackage{bigdelim, bigstrut}

%% [REC] Alternate vector arrows. Use the command \vv{} to get scaled
%% vector arrows.
\usepackage[h]{esvect}

%% [NEED] Some extensions to tabulars and array environments.
\usepackage{array}

%% [OPT] Postscript support via pstricks graphics package. Very
%% diverse applications.
%\usepackage{pstricks,pst-all}

%% [?] This seems to allow us to define some additional counters.
%\usepackage{etex}

%% [ADV] XY-Pic to typeset some matrix-style graphics
%\usepackage[all]{xy}

%% [OPT] This is needed to generate an index at the end of the
%% document.
%\usepackage{makeidx}

%% [OPT] Fancy package for source code listings.  The template text
%% needs it for some LaTeX snippets; remove/adapt the \lstset when you
%% remove the template content.
\usepackage{listings}
\lstset{language=TeX,basicstyle={\normalfont\ttfamily}}

%% [REC] Fancy character protrusion.  Must be loaded after all fonts.
\usepackage{microtype}

%% [REC] Nicer tables.  Read the excellent documentation.
\usepackage{booktabs}


%% Our layout configuration.  DO NOT CHANGE.
%% Memoir layout setup

%% NOTE: You are strongly advised not to change any of them unless you
%% know what you are doing.  These settings strongly interact in the
%% final look of the document.

% Dependencies
\usepackage{USIlogo}

% Turn extra space before chapter headings off.
\setlength{\beforechapskip}{0pt}

\nonzeroparskip
\parindent=0pt
\defaultlists

% Chapter style redefinition
\makeatletter

\if@twoside
  \pagestyle{Ruled}
  \copypagestyle{chapter}{Ruled}
\else
  \pagestyle{ruled}
  \copypagestyle{chapter}{ruled}
\fi
\makeoddhead{chapter}{}{}{}
\makeevenhead{chapter}{}{}{}
\makeheadrule{chapter}{\textwidth}{0pt}
\copypagestyle{abstract}{empty}

\makechapterstyle{bianchimod}{%
  \chapterstyle{default}
  \renewcommand*{\chapnamefont}{\normalfont\Large\sffamily}
  \renewcommand*{\chapnumfont}{\normalfont\Large\sffamily}
  \renewcommand*{\printchaptername}{%
    \chapnamefont\centering\@chapapp}
  \renewcommand*{\printchapternum}{\chapnumfont {\thechapter}}
  \renewcommand*{\chaptitlefont}{\normalfont\huge\sffamily}
  \renewcommand*{\printchaptertitle}[1]{%
    \hrule\vskip\onelineskip \centering \chaptitlefont\textbf{\vphantom{gyM}##1}\par}
  \renewcommand*{\afterchaptertitle}{\vskip\onelineskip \hrule\vskip
    \afterchapskip}
  \renewcommand*{\printchapternonum}{%
    \vphantom{\chapnumfont {9}}\afterchapternum}}

% Use the newly defined style
\chapterstyle{bianchimod}

\setsecheadstyle{\Large\bfseries\sffamily}
\setsubsecheadstyle{\large\bfseries\sffamily}
\setsubsubsecheadstyle{\bfseries\sffamily}
\setparaheadstyle{\normalsize\bfseries\sffamily}
\setsubparaheadstyle{\normalsize\itshape\sffamily}
\setsubparaindent{0pt}

% Set captions to a more separated style for clearness
\captionnamefont{\sffamily\bfseries\footnotesize}
\captiontitlefont{\sffamily\footnotesize}
\setlength{\intextsep}{16pt}
\setlength{\belowcaptionskip}{1pt}

% Set section and TOC numbering depth to subsection
\setsecnumdepth{subsection}
\settocdepth{subsection}

%% Titlepage adjustments
\pretitle{\vspace{0pt plus 0.7fill}\begin{center}\HUGE\sffamily\bfseries}
\posttitle{\end{center}\par}
\preauthor{\par\begin{center}\let\and\\\Large\sffamily}
\postauthor{\end{center}}
\predate{\par\begin{center}\Large\sffamily}
\postdate{\end{center}}

\def\@advisors{}
\newcommand{\advisors}[1]{\def\@advisors{#1}}
\def\@coadvisors{}
\newcommand{\coadvisors}[1]{\def\@coadvisors{#1}}
\def\@department{}
\newcommand{\department}[1]{\def\@department{#1}}
\def\@thesistype{}
\newcommand{\thesistype}[1]{\def\@thesistype{#1}}

\renewcommand{\maketitlehooka}{\noindent\USIlogo[2in]}

\renewcommand{\maketitlehookb}{\vspace{1in}%
  \par\begin{center}\Large\sffamily\@thesistype\end{center}}

\renewcommand{\maketitlehookd}{%
  \vfill\par
  \begin{flushright}
    \sffamily
    \@advisors\par
    \@coadvisors\par
    \@department, USI Lugano
  \end{flushright}
}

\checkandfixthelayout

\setlength{\droptitle}{-48pt}

\makeatother

% This defines how theorems should look. Best leave as is.
\theoremstyle{plain}
\setlength\theorempostskipamount{0pt}

%%% Local Variables:
%%% mode: latex
%%% TeX-master: "thesis"
%%% End:


%% Theorem environments.  You will have to adapt this for a German
%% thesis.
%% Theorem-like environments

%% This can be changed according to language. You can comment out the ones you
%% don't need.

\numberwithin{equation}{chapter}

%% German theorems
%\newtheorem{satz}{Satz}[chapter]
%\newtheorem{beispiel}[satz]{Beispiel}
%\newtheorem{bemerkung}[satz]{Bemerkung}
%\newtheorem{korrolar}[satz]{Korrolar}
%\newtheorem{definition}[satz]{Definition}
%\newtheorem{lemma}[satz]{Lemma}
%\newtheorem{proposition}[satz]{Proposition}

%% English variants
\newtheorem{theorem}{Theorem}[chapter]
\newtheorem{example}[theorem]{Example}
\newtheorem{remark}[theorem]{Remark}
\newtheorem{corollary}[theorem]{Corollary}
\newtheorem{definition}[theorem]{Definition}
\newtheorem{lemma}[theorem]{Lemma}
\newtheorem{proposition}[theorem]{Proposition}

%% Proof environment with a small square as a "qed" symbol
\theoremstyle{nonumberplain}
\theorembodyfont{\normalfont}
\theoremsymbol{\ensuremath{\square}}
\newtheorem{proof}{Proof}
%\newtheorem{beweis}{Beweis}


%% Helpful macros.
%% Custom commands
%% ===============

%% Special characters for number sets, e.g. real or complex numbers.
\newcommand{\C}{\mathbb{C}}
\newcommand{\K}{\mathbb{K}}
\newcommand{\N}{\mathbb{N}}
\newcommand{\Q}{\mathbb{Q}}
\newcommand{\R}{\mathbb{R}}
\newcommand{\Z}{\mathbb{Z}}
\newcommand{\X}{\mathbb{X}}

%% Fixed/scaling delimiter examples (see mathtools documentation)
\DeclarePairedDelimiter\abs{\lvert}{\rvert}
\DeclarePairedDelimiter\norm{\lVert}{\rVert}

%% Use the alternative epsilon per default and define the old one as \oldepsilon
\let\oldepsilon\epsilon
\renewcommand{\epsilon}{\ensuremath\varepsilon}

%% Also set the alternate phi as default.
\let\oldphi\phi
\renewcommand{\phi}{\ensuremath{\varphi}}


%% Make document internal hyperlinks wherever possible. (TOC, references)
%% This MUST be loaded after varioref, which is loaded in 'extrapackages'
%% above.  We just load it last to be safe.
\usepackage[linkcolor=black,colorlinks=true,citecolor=black,filecolor=black]{hyperref}





%% Document information
%% ====================

\title{On Qubit Routing in Quantum Computer Architectures}
\author{Mark Sobolev}
\thesistype{Master Thesis Proposal}
\advisors{Advisors: Prof.\ Dr.\ S. Wolf}
\coadvisors{Co-Advisors:  L. Laneve, W. Schober}
\department{Faculty of Informatics}
\date{\today}

\begin{document}

\frontmatter

%% Title page is autogenerated from document information above.  DO
%% NOT CHANGE.
\begin{titlingpage}

  \calccentering{\unitlength}
  \begin{adjustwidth*}{\unitlength-24pt}{-\unitlength-24pt}
    \vfill
    \maketitle
 	\end{adjustwidth*}
   \thispagestyle{empty}
    \newpage
    \vfill
    \noindent
    \begin{minipage}{.85\textwidth}
        \raggedright 
        \textit{The advisor(s) of this Master's Thesis Proposal confirm that the \underline{literature review} within this proposal, along with the \underline{scope} and \underline{content} of the proposed thesis topic, are sufficient for a master’s-level thesis defense in the Faculty of Informatics at Università della Svizzera italiana..}
        \vfill
        \vspace{10em}
    	\noindent\hfill\makebox[2.5in]{\hrulefill} \\ 
    	\noindent\hfill\textit{Signature of Advisor(s) \& Date}
    \end{minipage}

\end{titlingpage}


%% The abstract of your thesis.  Edit the file as needed.
\begin{abstract}

The basis of all the modern quantum algorithms was outlined by Deutsch \cite{deutschQuantumComputationalNetworks1989}. It presented a modelling paradigm for all the quantum algorithms, later proven through the quantum Turing machine equivalence conjecture by Yao \cite{chi-chihyaoQuantumCircuitComplexity1993}. The creation of a quantum computer architecture nowadays is acting as a basic toolset for the projection and analysis of all the quantum computation algorithms. However, the realistic hardware architecture is highly different from the traditional computational model -- which is de facto assumed by Deutsch, i.e. von Neumann architecture \cite{vonneumannFirstDraftReport1993}. This departure in particular is happening because the qubits in the layout don't have the equivalent connectivity properties, i.e. not all the physical qubits in the circuit are made equal (see, for example, an informal discussion of IBM quantum architecture connectivity \cite{IBMQuantumHeavy}). Thus, an additional constraint satisfaction at the compilation step is required, termed "qubit routing". We will formalise the process of quantum circuit conversion to the qubit interaction graph, combining time- and spatial-related information. Then we aim to explore the possibilities of the optimal placement of CNOT gates (and potential insertion of SWAP gates) in the architecture-agnostic computation graph. After that, we will take the constraints of the different quantum computer architectures and adapt our routing/mapping algorithms to those, presenting an additional set of graphs. Finally, we will port the results of the routing to the main quantum circuit compilers, e.g. qiskit, and evaluate those on the main benchmarks for the selected quantum architectures.
\end{abstract}


%% TOC with the proper setup, do not change.
\cleartorecto
\tableofcontents
\mainmatter

%% Your real content!
% Some commands used in this file
\newcommand{\package}{\emph}

\chapter{Introduction}

Qubit routing is a problem which emerged with the rise of NISQ devices and is a result of quantum hardware restrictions. The go-to model for planning and development of quantum algorithms was created by Deutsch \cite{deutschQuantumComputationalNetworks1989}. Deutsch designed, having a quantum Turing machine in mind, a quantum circuit model, which is represented by a set of wires (each of them represents a quantum bit ), and a set of quantum gates, which are ordered execution-wise. Every quantum gate can employ one or more wires (without any location requirements — every combination of wires can participate in any gate). The only model restriction is that — qubit can’t be engaged in multiple operations, represented by quantum gates, at the same time. And from this unbounded representation comes the qubit routing problem — quantum hardware is not equivalent (and probably can’t) to the circuit model. In modern quantum hardware, qubits can interact in multi-qubit gates only with a fixed set of physical qubits. However unwanted, this problem could be dealt with by the SWAP gate insertion, which effectively swaps two qubits; and if we assume all-to-all qubit connectivity (more formal definition would be — if the hardware topology graph is represented by one completely connected component), then this whole problem is boiled down to the efficient swap gates insertion, that’s why sometimes “qubit routing” problem is called “swap insertion”. Our work aims to overview the field of qubit routing, to present a study of a problem, and to formulate a solution to it.
\chapter{Related works}

As a good introduction to the problem, we present you a detailed review of an important paper by Cowtan \cite{cowtanQubitRoutingProblem2019}. Cowtan formalised the qubit routing problem, and suggested an heuristic solution to it. 

\section{Cowtan's solution}

Cowtan and the team have presented a way to abstract mathematically the qubits routing, and then use a basic heuristic to solve it. They start by stating that the current quantum architectures are quite distant from the traditional von Neumann architecture, since qubits usually need to be neighbouring to perform operations on them. This is done by inserting swap operations to transfer qubits to different places of the circuit. This problem is known to be infeasible, and there are some go-to solutions available, the authors state. In the tket (Cambridge quantum compiler), both the translation and the routing to the given set of hardware qubits are implemented. 

\subsection{Problem statement}

The authors then represent a quantum computer as a graph, where the nodes are qubits and edges are possible interactions of the qubits. The quantum circuit model by default is a fully connected graph; thus, they needed to produce a quantum circuit respecting the hardware constraints. As an example of a ring hardware connectivity graph, it is shown that it is beneficial to slice the circuit by the timesteps first. Then the authors suggest mapping the nodes to qubits in a smart way. Finally, for every timestep, you check if there is a two-qubit gate on non-connected qubits. In such a case, you insert a swap gate chain to get to the required neighbouring pair. 

The authors state that the task is bounded by  $n^2$ swap operations, going from one combination to any other. A recent study has shown that it can be decreased to $\log n$ \cite{herbertDepthOverheadIncurred2020}. 
But it is obvious that it is needed to map the circuit globally and this one was per one timestep. 

Another mentioned problem type is the fact that swap gate, being more or less an elementary operation, can be problematic on some hardware architectures. In example, the superconducting ones only have one elementary building block, so an ansatz is required. Also, for IBM architectures sometimes  only one directional CNOT gates are available.

\subsection{Algorithm overview}

After that, the routing process is described by the authors. They start with an incomplete initial mapping of hardware qubits, and then go through all the timesteps. In a given timestep, if they approach yet not mapped qubit — they add it to a nearest neighbour of its two-qubit gate vis-\`{a}-vis. If after this operation all the constraints are satisfied, they proceed to the next slice, otherwise, they need to start building the chain of SWAP gates.


They then present an overview of the algorithm, namely: 1) the timestep decoposition, 2) an initial placement determination, 3)the routing itself, and 4) a cleanup phase. 

\subsection{Circuit slicing}

The cutting of the circuit into timesteps is a greedy algorithm, which slices the circuit by the two-qubit gates. If a two-qubit gate requires the qubit to be present, which is already participating in the current timestep, then a new timestep is formed. Authors also present a density metric, which represents the fraction --- how many two-qubit gates out of the maximum possible one happen in the timestep. They then remark that they are okay with sparsity, but there is a possibility of constraining the sparsity by this density metric.

\subsection{Initial mapping}

The next phase is initial mapping: it is suggested to be constructed by a heuristic analysis, iterating through timesteps, an edge between two qubits only if they interact in this timestep, and for both of them, it is less than the third interaction in the whole circuit. As a result of this operation, authors claim that there can be either a ring or a line shape of the initial-mapping graph. The ring is broken then by removing a random connection. That way of initial mapping guarantees that most of the first two interaction gates (except that one which was chosen to break the ring) are happening without any swaps. After this, you select a subgraph of the hardware connectivity graph with the highest average degree and lowest diameter, and then map your mapping as a one long line, starting from the highest degree hardware qubit, and moving to the next highest degree qubit from there. If the initial mapping is impossible on the given architecture, you can split the mapping line into several line segments.

\subsection{Routing algorithm}

The routing algorithm also runs step by step. It is suggested to provide the initial mapping and the sliced circuit as an input to the algorithm. If there are two-qubit gate qubits in the timestep, which are yet to be mapped, they are just mapped to a nearest mapped gate neighbour. All the one-qubit gates and the executable two-qubit gates are immediately removed from the sliced circuit. Then the swap-distance vectors are constructed. Authors suggest starting from the set of all available swaps in the circuit; then, you take all swaps that maximally reduce the distance to the circuit being executable. They repeat the construction until the path is unique, or just select a random one in a cutoff fashion. Authors argue that the diameter of the graph should strictly decrease under this construction.

Sometimes the algorithm gets stuck due to high symmetricity of the hardware connectivity graph:  then it is suggested to try to have two-swaps in the construction phase; or even a brute force full longest two-qubit path once can be applied.

In practice, they do the slicing dynamically, so no pre-slicing of the circuit is needed. Also, at the end of the routing, a small clean-up pass happens either to get a hardware mapping or to remove excessive gates.  





%%% Local Variables:
%%% mode: latex
%%% TeX-master: "thesis"
%%% End:

\chapter{Thesis Plan}


In short, this master's thesis should present the problem of qubit routing. First, we expect to explain the problem by listing the connectivity requirements of the modern hardware, and then showing the existing solutions to the problem. Then we expect to outline the most prominent ways of the algorithmic analysis of the problem, the main ways of problem trivialisation (i.e. reduction to a known subproblem) and the known heuristic solutions to the problems, especially those with the components of the serious theoretical treatment. As a main body of work, we aim to present the terminological overview of the subfield, present a complete mathematical model of the problem,  and then a solution to this mathematical construction. A final step would include an evaluation of the presented solution, and the benchmarking and comparative analysis in the contemporary state of the field. 

\section{Introduction}

In this part we would like to discover all the available solutions and analyses in the field, as well as the state of the hardware connectivity.

\subsection{A Review of Quantum hardware}
In this section, we would like to present the known hardware implementations of the quantum computers; namely, the topology of the connectivity properties is of interest. We would like also to present some insights about possible modifications of the quantum qubit routing problem --- see the neutral atom \cite{wangAtomiqueQuantumCompiler2024}, or race track \cite{mosesRaceTrackTrappedIon2023} trapped ion architecture, which are an unorthodox approach to the problem if we'd compare it with other contenders (the whole notion of the transformation of the qubit connectivity space is problematic for our suggested analysis). Thus, we'd like to convey as much information as possible on the hardware implementation topics. Well-known quantum architecture topologies include the IBM hexagonal \cite{IBMQuantumHeavy} and Google bowtie grid \cite{aruteQuantumSupremacyUsing2019}. The additional complexity of this part arises from the fact that the implementation is often a part of a trade secret, and only singular examples of such unveil exist. As an output of this section, we would like to present the extensive review of the most prominent quantum hardware implementations. 


\subsection{Qubit routing approaches}

The problem of qubit routing is natural to the way modern NISQ devices are implemented. Thus, a lot of possible solutions to the problem were created in the last decade. The first of the modern outlooks and attempts of formalisation of the problem is the \cite{cowtanQubitRoutingProblem2019}, presented in detail in the related work section. It presents both a way to analyse the problem (both token swapping and graph isomorphism ways of thinking), a way to solve the problem, and a set of benchmarks. In this section, we would like to present all the aforementioned elements of the successful work, collecting the information about the methods of analysis, ways of solving the problem and benchmarking. We would like to deeply explore the SABRE family of algorithms \cite{liTacklingQubitMapping2019}--- which are considered state-of-the-art, and used in the IBM Qiskit compiler set. A central paper in the algorithmic sense is the \cite{itoAlgorithmicTheoryQubit2023}, which provides a rich example of how to model the qubit routing problem, with a huge set of reasonable conjecture results. Finally, there are some algorithms designed specifically for an architectural paradigm, like \cite{willeMQTQMAPEfficient2023}. It would be good to outline their ideas and present a compatibility study of them. A good result of this section would be to create a thorough understanding of the modern qubit routing solving approaches.


\section{Related work}

This section is devoted to more particular analysis of solutions to the problem.

\subsection{Qubit routing reductions}
One of the most prominent ways of solving the problem efficiently is reducing it to a well-studied type of problem, and then use an existing problem solver, or enhance the analysis by this inclusion. Some attempts of doing so for the qubit routing problem were made by \cite{brierleyEfficientImplementationQuantum2016}; some other papers include reduction to MaxSat or token swapping. Some alternative well-known graph structures were suggested: cyclic butterfly, hypercube and others. This section aims to provide an overview of trivialisations, their pros and cons, effects, and possible modifications or reductions to be made.  

\subsection{Qubit routing analysis}


In this section, the most looked upon part in the whole qubit routing task will be overviewed. The algorithmic analysis of the problem suggests analysing various problem formulations, problem solutions and the characteristics of these solutions. An excellent example of such analysis is presented in \cite{itoAlgorithmicTheoryQubit2023}. We aim to collect and entwine the analyses (often incomplete) and present a set of possible problem formulations, useful conjectures and lemmas. 

\subsection{Qubit routing solutions}

This section will outline the most important heuristic and algorithmic approaches to solving the routing problem. As even the problem formulation is different in many sources, this will require splitting the existing solutions into hierarchical components (responsible for different parts of the problem) and matching those together. A good overview of the contemporary is available at (\cite{barnesSurveyQubitRouting2023}). There are many sources of confusion in the literature, for example, some are differentiating so-called qubit mapping and qubit routing (that wouldn’t be a problem, but this sometimes acts as a complexity well — analysing only one part, but moving a lot of steps in the other). Most of the approaches follow the timestep slicing as a main technique, and dynamic analysis of each timestep, like \cite{cowtanQubitRoutingProblem2019}. However, there are also machine learning solutions (\cite{pozziUsingReinforcementLearning2020}), reductionist solutions (\cite{molaviQubitMappingRouting2022}) and many other groups. An ideal outcome of this section would be a solutions taxonomy, their shared characteristics and different approaches.

\section{Method}

This part will be related to our approach to solving the qubit routing problem.

\subsection{Terminological clarity}

This section would provide an overview of the employed terminology, marking the start of creative contributions to the field. The terminological situation in the qubit routing (as you’ve probably noted in the previous section) is often a source of confusion. What is \cite{willeMQTQMAPEfficient2023} names allocation, is routing in \cite{cowtanQubitRoutingProblem2019}, swap insertion in \cite{itoAlgorithmicTheoryQubit2023}  etc. The problem statements (e.g. some include connectivity graphs in statement, some just a general overview), operations definitions (do we include the one-qubit gates, how do we define a timestep ) — are only a small fraction of terminological problems. The ideal result of this section would be a simple guideline for qubit routing problems, terminology and interface between different schools and research groups. 

\subsection{Problem modelling}

This chapter would outline our understanding of the qubit routing problem, providing a robust mathematical description of the problem. We plan to take the best of both worlds --- theoretical and practical --- and formulate the problem most clearly and concisely. Finding the theoretical representation of the problem is the demand of the analytical part of our work, and finding an effective practical translation of it is the demand of the practical part. Therefore, this part is also amplified since it lays the foundation of future chapters: solving the problem, evaluating the result and others. As a small insight into the planned approach, we had an idea of using Cayley diagram \cite{akersGrouptheoreticModelSymmetric1989} representation of the problem as the theoretical basis of the analysis. However powerful, it is also exponential in the qubit count; thus, for actually solving the problem, it can be rendered highly impractical. Maybe some composite solution would be possible, however, the other problem would be --- if we hurt the analytical stiffness of the problem statement, we might lose all the expressive power of our framework. The ideal output of this chapter would be a robust mathematical statement of the problem and its practical application.  

\subsection{Problem analysis}

This section aims to present an all-sided analysis of the qubit routing problem. First part of it — algorithmic analysis, similar to the approach suggested by \cite{itoAlgorithmicTheoryQubit2023}. We’d like to analyse the approach in the lenses of our problem statement, optimally obtaining the bounds for the algorithmic complexity, and analysis of the problem hardness for different classes of hardware and circuit layouts. The possible reductions of the problem will also be considered (as in MaxSat \cite{molaviQubitMappingRouting2022}), as a path to use a bigger class of universal algorithmic solvers. An ideal result of this chapter would be complete results about the algorithmic properties of our problem model and possible reductions of the problem to other problems.

\subsection{Solving the problem}

This is the most important part of the whole thesis. Here we would like to outline the algorithm for solving the problem stated and analysed in the previous chapters. As it was outlined in the previous sections, the approach to problem solving, although dependent on the analysis, could also be heuristic, reductionist or based on a problem extension. It would be good to present multiple approaches, as all of the aforementioned ways of solving have their differences, e.g. heuristic methods are usually robust and easy to implement, see \cite{zouLightSABRELightweightEnhanced2024}. If we obtain a new reduction model, we’ll use an existing solver and explain why we used it. If it will be an extension of a problem, e.g. machine learning approach, then we’ll show the intuition behind the model, and the cost of this extension. Nevertheless, an ideal result of this chapter would be at least one approach to solving the presented problem, with an analysis (if possible) and its motivation.

\section{Evaluation and discussion}

The last planned part is an evaluation part. We want to analyse our problem solution from multiple sides. First, a comparison of analytical results with state-of-the-art solutions would be needed, including time complexity, scaling analysis, etc. Second, a practical benchmarking should be applied, a set of useful benchmarks could be found, for example in \cite{cowtanQubitRoutingProblem2019}. The ablation study of the solution of the problem is also desired, to see which parts affect the performance of the algorithm; also, verification of theoretical results is required. An analysis of the comparative performance could be a good addition to this section, together with a scalability study. An ideal output of this section would be plots of scalability, and a table of benchmark performance. 




\backmatter

\bibliographystyle{plain}
\bibliography{refs}

\includepdf[pages={-}]{declaration-originality.pdf}

\end{document}
